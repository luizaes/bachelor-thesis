Metaheuristics are stochastic algorithms, therefore, randomization is one of its fundamental principles. Randomization allows the algorithms not only to be able to escape from local minimums or maximums, performing a global search, but also helps in local searches in the neighborhood of the current best solution. Taking this into consideration, we have several randomization methods that can be used in these algorithms, such as the Uniform, Gaussian, and Cauchy distributions, as well as chaotic maps, such as Logistic and Kent. However, most studies do not analyze the best distribution that can be used. Therefore, this paper focuses on the analysis of the impact that different probability distributions have on some metaheuristics. The metaheuristics selected for this analysis were the Sine Cosine Algorithm (SCA) and the Self-Adaptive Differential Evolution (jDE), chosen due the simplicity of their implementation. The distributions used were the Gaussian distribution and the Logistic chaotic map, in addition to the Uniform distribution, used for comparison. The algorithms were applied to two different types of problems: initially, on a selected set of benchmark functions, and then on a real-world problem known as Protein Structure Prediction in the 2D-AB model. The results of the experiments show that for benchmark functions the Uniform distribution proves to be the best randomization method option for SCA and jDE. However, when applied to a real-world problem, SCA shows better results with the Logistic map. Moreover, SCA, when adapted with a Gaussian distribution with a variable standard deviation in its main parameters, obtains results superior to the original algorithm.

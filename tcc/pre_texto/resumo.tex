Meta-heurísticas são algoritmos estocásticos, portanto, a randomização é um dos seus princípios fundamentais. A randomização permite que o algoritmo não só seja capaz de escapar de mínimos ou máximos locais, fazendo uma busca global, como também ajuda em buscas locais na vizinhança da atual melhor solução. Levando isso em consideração, temos a disposição diversos métodos de randomização que podem ser utilizados nos algoritmos, como a distribuição Uniforme, Gaussiana, e de Cauchy, além dos mapas caóticos, como o Logístico e de Kent. Entretanto, grande parte dos estudos não fazem uma análise de qual é a melhor distribuição para se utilizar. Por isso, este trabalho foca na análise do impacto que diferentes distribuições probabilísticas possuem sobre algumas meta-heurísticas. As meta-heurísticas selecionadas para realização desta análise foram o Algoritmo Seno e Cosseno (SCA) e a Evolução Diferencial Auto-adaptativa (jDE), escolhidas devido a simplicidade de implementação. As distribuições utilizadas foram a Gaussiana e o mapa caótico Logístico, além da distribuição Uniforme, para comparação. Os algoritmos foram aplicados em dois tipos de problemas diferentes: inicialmente, em um conjunto selecionado de funções \textit{benchmark} e depois em um problema do mundo real conhecido como Predição da Estrutura de Proteínas no modelo 2D-AB. Os resultados dos experimentos mostram que para as funções \textit{benchmark} a distribuição Uniforme apresenta-se como a melhor opção de método de randomização para o SCA e o jDE. Entretanto, quando aplicados em um problema do mundo real, o SCA mostra melhores resultados com o mapa Logístico. Ademais, o SCA, quando adaptado com uma distribuição Gaussiana com desvio-padrão variável em seus principais parâmetros, obteve resultados superiores ao algoritmo original.
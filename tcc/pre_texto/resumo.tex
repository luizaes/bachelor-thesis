Meta-heurísticas são algoritmos estocásticos, portanto, a randomização é um dos seus princípios fundamentais. A randomização permite que o algoritmo não só seja capaz de escapar de mínimos ou máximos locais, fazendo uma busca global, como também ajuda em buscas locais na vizinhança da atual melhor solução. Levando isso em consideração, temos a disposição diversos métodos de randomização que podem ser utilizados nos algoritmos, como a distribuição Uniforme, Gaussiana, e de Cauchy, além dos mapas caóticos, como o Logístico e de Kent. Entretanto, grande parte dos estudos não fazem uma análise de qual a melhor distribuição a se usar. Por isso, este trabalho foca na análise do impacto que diferentes distribuições probabilísticas possuem sobre algumas meta-heurísticas.